\documentclass[10pt]{beamer}
\usetheme{Singapore}
\usepackage{xunicode}
\usepackage{xecyr}
\usecolortheme{beaver}
\usepackage{fontspec}
\setsansfont{FreeSans}
\setmonofont{DejaVuSansMono}
\usepackage[russian]{babel}

\title{Транзакции, SQL и PostgreSQL}
\author{Александр Неганов}
\date{\today}
\institute{Selectel}

\begin{document}

\frame{\titlepage}

\frame{
  \frametitle{Транзакции}

  Что такое транзакция?

  \begin{itemize}
    \item<2-> Версия состояния базы данных
    \item<3-> Все или ничего
    \item<4-> Изоляция
  \end{itemize}
}

\frame{
  \frametitle{Изоляция транзакций}
  
  ``Феномены`` (проблемы) которые могут возникнуть:

  \begin{itemize}
    \item<2> Потерянное обновление (lost update)
    \item<3-> Потерянное обновление (lost update) - в SQL стандартах
      такого не бывает
    \item<4-> Грязное чтение (dirty read)
    \item<5-> Неповторяющееся чтение (non-repeatable read)
    \item<6-> Фантомное чтение/фантомная вставка (phantom read)
  \end{itemize}
}

\frame{
  \frametitle{Уровни изоляции транзакций}

  Стандарт SQL определяет четыре уровня изоляции транзакций:

  \begin{itemize}
    \item<2-> \texttt{READ UNCOMMITTED} (dirty read, non-repeatable read,
      phantom read)
    \item<3-> \texttt{READ COMMITTED} (non-repeatable read, phantom read)
    \item<4-> \texttt{REPEATABLE READ} (phantom read)
    \item<5-> \texttt{SERIALIZABLE} - никакие эффекты невозможны, параллельное
      выполнение - то же самое что и последовательное.
  \end{itemize}
 
}

\frame{
  \frametitle{Транзакции, как их понимает PostgreSQL 9.1}

  \begin{itemize}
    \item<2-> Транзакции могут видеть данные только подтвержденных (committed) транзакций ...
    \item<3-> ... \texttt{READ UNCOMMITTED} то же что и \texttt{READ COMMITTED}
    \item<4-> ... Фантомное чтение само по себе невозможно
    \item<5-> \texttt{UPDATE} / \texttt{DELETE} лочит изменяемые строки до окончания транзакции
    \item<6-> ... что может привести к dead lock'ам
    \item<7-> ... которые хорошо определяются
    \item<8-> Ошибка сериализации (40001)
    \item<9-> \texttt{REPEATABLE READ} - ``слепок`` состояния БД на начало транзакции + ошибки сериализации
    \item<10-> \texttt{SERIALIZABLE} - проверка зависимостей данных, ошибки сериализации
  \end{itemize}

}

\frame{
  \frametitle{READ COMMITTED / READ UNCOMMITTED}

  Caмый слабый уровень изоляции транзакций

  \begin{itemize}
    \item<2-> Уровень изоляции по-умолчанию
    \item<3-> Каждый запрос видит состояние БД на момент начала запроса.
    \item<4-> \texttt{UPDATE} / \texttt{DELETE} лочит строки, удовлетворяющие \texttt{WHERE},
      перепроверяет и изменяет.
    \item<5-> Могут возникать странные, очень странные эффекты.
  \end{itemize}

}

\frame{
  \frametitle{REPEATABLE READ}

  \begin{itemize}
    \item<2-> ``Слепок`` состояния БД на начало транзакции
    \item<3-> \texttt{UPDATE} / \texttt{DELETE} пытается лочить строки, удовлетворяющие WHERE, и, возможно, выкидывает ошибку сериализации
    \item<4-> ... к которой нужно быть готовым
    \item<5-> До 9.1 выбирался при указании \texttt{SERIALIZABLE}
  \end{itemize}


}
\frame{
  \frametitle{SERIALIZABLE}

  Самый строгий уровень изоляции транзакций

  \begin{itemize}
    \item<2-> ``Слепок`` состояния БД на начало транзакции
    \item<3-> Проверка зависимостей по данным между транзакциями, ошибка сериализации
    \item<4-> Не добавляет больше локов чем \texttt{REPEATABLE READ}
    \item<5-> ... но разумеется, есть overhead
    \item<6-> Sequental scan затрагивает все строки -> больше ошибок сериализации
    \item<7-> ... но зато строго повторяемый результат
    \item<8-> ... но пока нет поддержки для Hot Standby
  \end{itemize}

}

\frame{
  \frametitle{Транзакции: выводы}

  \begin{itemize}
    \item<2-> Простые транзакции - \texttt{READ COMMITTED}, но с осторожностью
    \item<3-> Отсутствие зависимостей по данным - \texttt{REPEATABLE READ}
    \item<4-> Надежно, но без Hot standby и можно медленно - \texttt{SERIALIZABLE}
  \end{itemize}
}

\frame{
  \frametitle{SQL}

  Создан в 1974, последняя ревизия стандарта - 2011 год.

  \begin{itemize}
    \item<2-> Стандарт не является полностью свободно/бесплатно доступным.
    \item<3-> Стандарт не покрывает всех нужд, существуют свои расширения
    \item<4-> ... тем не менее, поддержка XML там есть.
    \item<5-> Отличающиеся реализации стандарта разными СУБД
    \item<6-> \texttt{NULL}, \texttt{IS NULL}, constraints, аргумент функций
    \item<7-> Один из самых понятных DSL, по мнению независимых экспертов
  \end{itemize}
}

\frame{
  \frametitle{Вопросы?}
}
\end{document}
